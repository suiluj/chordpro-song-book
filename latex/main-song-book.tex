%
% Body text font is Palatino!
%

% no empty pages when starting chapter on twoside https://tex.stackexchange.com/a/46196

\documentclass[a5paper,pagesize,10pt,bibtotoc,pointlessnumbers,
normalheadings,BCOR=10mm,DIV=calc,openany,twoside=true]{scrbook}

\newcommand*{\includepdfsong}[4]{
	
	\includepdf[width=\textwidth,offset=3.5mm 0,pages=-,pagecommand={\thispagestyle{plain}},addtotoc={
		1,section,1,#1,sec:song-#2-#3}] 
	{#4}
	
}

%custom area or better presets: https://stackoverflow.com/a/62811606/5628238

% DIN A5: 148 x 210 mm
\areaset[current]{130.00mm}{180mm}

% twoside, openright
%\KOMAoptions{DIV=last}

%\usepackage{showframe}

\usepackage{trajan}

\usepackage{blindtext}
 
\usepackage[ngerman]{babel}
\usepackage[utf8]{inputenc}
\usepackage[T1]{fontenc}

\usepackage[scaled]{helvet}

\usepackage[babel,german=guillemets]{csquotes}

\usepackage{pdfpages}

%\usepackage[bottom=6em]{geometry}

\usepackage{imakeidx}
\makeindex[options=-s indexstyle.ist,columns=1,title=Alphabetische Songliste, intoc]
% use imakeindex style files: https://www.overleaf.com/learn/latex/indices#Using_style_files
% custom line dotted: https://tex.stackexchange.com/questions/132465/right-aligning-pagenumbers-in-index-with-imakeindex

% für index tools befehle make index ausführen und nochmal generieren lassen:
% https://tex.stackexchange.com/a/98215
% https://www.namsu.de/Extra/pakete/Makeidx.html
% alternative imakeidx package: https://ftp.rrze.uni-erlangen.de/ctan/macros/latex/contrib/imakeidx/imakeidx.pdf

%\usepackage[sc]{mathpazo}
%\linespread{1.05} 

\setlength\parindent{0pt}

% https://tex.stackexchange.com/questions/73862/how-can-i-make-a-clickable-table-of-contents
\usepackage[hidelinks]{hyperref}
\hypersetup{
%	colorlinks,
%	citecolor=black,
%	filecolor=black,
%	linkcolor=black,
%	urlcolor=black
}

%\usepackage{verbatim} % for comments
%\usepackage{listings} % for comments

%\setlength{\parindent}{10pt}
%\setlength{\parskip}{1.4ex plus 0.35ex minus 0.3ex}
%\setlength{\parskip}{1.4ex plus 0.35ex minus 0.3ex}

%\usepackage{blindtext}
%\newcommand{\q}[1]{>>\textit{#1}<<}

\newcommand{\VAR}[1]{}
\newcommand{\BLOCK}[1]{}

%\newcommand*{\songsectionspath}{chapters/song-sections}

\title{DePoLi}   
\author{Gemeinschaft Junges Ermland} 
\date{\today} 


\renewcommand\familydefault{\sfdefault}
\begin{document}


%=========================================
\begin{titlepage}
		\centering{
			{\fontsize{40}{48}\selectfont 
			DePoLi}
		}\\
			
		\vspace{10mm}
		\centering{\Large{Gemeinschaft Junges Ermland}}\\
		
		\vspace{10mm}
		\centering{\small{(hier kommt natürlich noch ein cooles Cover hin)}}\\
		
		\vspace{\fill}
		\centering \large{\today}
\end{titlepage}


%=========================================
%\newpage{}
%\thispagestyle {empty}
%
%\vspace*{2cm}
%
%\begin{center}
%	\Large{\parbox{10cm}{
%		\begin{raggedright}
%		{\Large 
%			\textit{Do what you think is interesting, 
%			do something that you think is fun and worthwhile, 
%			because otherwise you won’t do it well anyway.}
%		}
%	
%		\vspace{.5cm}\hfill{---Brian W. Kernighan}
%		\end{raggedright}
%	}
%}
%\end{center}

\tableofcontents


\newpage


%%https://www.namsu.de/Extra/pakete/Makeidx.html
%% not working at beginning of document: https://tex.stackexchange.com/questions/193460/unable-to-print-index-at-the-beginning-of-the-document
%\printindex
%\newpage

\chapter{Info}

In diesem Liederbuch werden die einzelnen Lieder als PDFs automatisch per Script eingebunden.\\

\blindtext[5]

%Sing mal \autoref{sec:song-VolksliederRockPopDeutsch-99LuftballonsNena}


%Sing mal \hyperref[sec:song-VolksliederRockPopDeutsch-99LuftballonsNena]{99 Luftballons}! :)




\BLOCK{ for section in sections }
\input{chapters/song-sections/\VAR{section.title_include_path_and_ref}}

\BLOCK{ endfor }



%=========================================
%\blinddocument


%=========================================
%\begin{comment}
%Just some notes, not visible in pdf.
%\end{comment}

%https://www.namsu.de/Extra/pakete/Makeidx.html
% or imakeidx
\printindex

\end{document}